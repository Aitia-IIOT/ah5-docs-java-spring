\documentclass[a4paper]{arrowhead}

\usepackage[yyyymmdd]{datetime}
\usepackage{etoolbox}
\usepackage[utf8]{inputenc}
\usepackage{multirow}
\usepackage{hyperref}

\renewcommand{\dateseparator}{-}

\setlength{\parskip}{1em}

%% Special references
\newcommand{\fref}[1]{{\textcolor{ArrowheadBlue}{\hyperref[sec:functions:#1]{#1}}}}
\newcommand{\mref}[1]{{\textcolor{ArrowheadPurple}{\hyperref[sec:model:#1]{#1}}}}
\newcommand{\pdef}[1]{{\textcolor{ArrowheadGrey}{#1\label{sec:model:primitives:#1}\label{sec:model:primitives:#1s}\label{sec:model:primitives:#1es}}}}
\newcommand{\pref}[1]{{\textcolor{ArrowheadGrey}{\hyperref[sec:model:primitives:#1]{#1}}}}

\newrobustcmd\fsubsection[3]{
  \addtocounter{subsection}{1}
  \addcontentsline{toc}{subsection}{\protect\numberline{\thesubsection}operation \textcolor{ArrowheadBlue}{#1}}
  \renewcommand*{\do}[1]{\rref{##1},\ }
  \subsection*{
    \thesubsection\quad
    operation
    \textcolor{ArrowheadBlue}{#1}
    (\notblank{#2}{\mref{#2}}{})
    \notblank{#3}{: \mref{#3}}{}
  }
  \label{sec:functions:#1}
}
\newrobustcmd\msubsection[2]{
  \addtocounter{subsection}{1}
  \addcontentsline{toc}{subsection}{\protect\numberline{\thesubsection}#1 \textcolor{ArrowheadPurple}{#2}}
  \subsection*{\thesubsection\quad#1 \textcolor{ArrowheadPurple}{#2}}
  \label{sec:model:#2} \label{sec:model:#2s} \label{sec:model:#2es}
}

\begin{document}

%% Arrowhead Document Properties
\ArrowheadTitle{DataModelTranslator Application System}
\ArrowheadType{System Description}
\ArrowheadTypeShort{SysD}
\ArrowheadVersion{5.1.0}
\ArrowheadDate{\today}
\ArrowheadAuthor{Tamás Bordi}
\ArrowheadStatus{DRAFT}
\ArrowheadContact{tbordi@aitia.ai}
\ArrowheadFooter{\href{www.arrowhead.eu}{www.arrowhead.eu}}
\ArrowheadSetup
%%

%% Front Page
\begin{center}
  \vspace*{1cm}
  \huge{\arrowtitle}

  \vspace*{0.2cm}
  \LARGE{\arrowtype}
  \vspace*{1cm}

  %\Large{Service ID: \textit{"\arrowid"}}
  \vspace*{\fill}

  % Front Page Image
  %\includegraphics{figures/TODO}

  \vspace*{1cm}
  \vspace*{\fill}

  % Front Page Abstract
  \begin{abstract}
    This document provides system description for the \textbf{DataModelTranslator Application System}.
  \end{abstract}

  \vspace*{1cm}

 \end{center}

\newpage
%%

%% Table of Contents
\tableofcontents
\newpage
%%

\section{Overview}
\label{sec:overview}
\color{black}
This document provides a \textbf{general description} of the DataModelTranslator Application System, which enables translation between different data models.

The rest of this document is organized as follows.
In Section \ref{sec:use}, we describe the intended usage of the system.
In Section \ref{sec:properties}, we describe fundamental properties
provided by the system.
In Section \ref{sec:delimitations}, we describe delimitations of capabilities
of the system.
In Section \ref{sec:services}, we describe the abstract services produced by the system.
In Section \ref{sec:security}, we describe the security capabilities
of the system.


\subsection{How This System Is Meant to Be Used}
\label{sec:use}

DataModelTranslator is a provider system that has the capability to integrating into a translation bridge initiated by the TranslationManager Support System. The integration is achieved by providing the \textbf{dataModelTranslation} service, that is consumed by the \textit{InterfaceTranslator} application systems during an actual translation between a consumer and a provider. 

Alternatively, its translation function might be used independently from a translation bridge. 

\subsection{System functionalities and properties}
\label{sec:properties}

\subsubsection {Functional properties of the system}
DataModelTranslator solves the following needs to fulfill the translation functionality.

\begin{itemize}
    \item Enables other systems to start a data model translation task.
    \item Enables other systems to check the statuses of existing data model translation tasks.
    \item Enables other systems to get the results of existing data model translation tasks.
    \item Enables other systems to abort existing data model translation tasks.
\end{itemize}

Data models are expected to be referenced by a unique ID that defines both a specific format (e.g., JSON, XML) and its associated semantics.

\textbf{Note:} This specification does not define the exact data models to be handled. DataModelTranslator provider instances can differ in the data models they support, but they are always required to expose their capabilities through the common \textbf{dataModelTranslation} service.

\subsubsection {Non functional properties of the system}
-

\subsubsection {Data stored by the system}

The DataModelTranslator shall not persist any data about individual translation processes, but it may store data necessary for the translation process itself.

\subsection{Important Delimitations}
\label{sec:delimitations}

If data model identifiers are not available within a Local Cloud, then data model translations are not possible.

\section{Services produced}
\label{sec:services}

\phantomsection
\msubsection{service}{dataModelTranslation}

The purpose of this service is to provide functionality for translating between
one or more exact data model pairs.

\section{Security}
\label{sec:security}

DataModelTranslator provider instances shall support all security levels enabled by a given Local Cloud.
Instances may limit their access to specific InterfaceTranslator providers and require encrypted communication.

\newpage

\bibliographystyle{IEEEtran}
\bibliography{bibliography}

\newpage

\section{Revision History}
\subsection{Amendments}

\noindent\begin{tabularx}{\textwidth}{| p{1cm} | p{3cm} | p{2cm} | X | p{4cm} |} \hline
\rowcolor{gray!33} No. & Date & Version & Subject of Amendments & Author \\ \hline

1 & YYYY-MM-DD & \arrowversion & & Xxx Yyy \\ \hline
\end{tabularx}

\subsection{Quality Assurance}

\noindent\begin{tabularx}{\textwidth}{| p{1cm} | p{3cm} | p{2cm} | X |} \hline
\rowcolor{gray!33} No. & Date & Version & Approved by \\ \hline

1 & YYYY-MM-DD & \arrowversion  &  \\ \hline

\end{tabularx}

\end{document}