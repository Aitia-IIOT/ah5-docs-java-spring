\documentclass[a4paper]{arrowhead}

\usepackage[yyyymmdd]{datetime}
\usepackage{etoolbox}
\usepackage[utf8]{inputenc}
\usepackage{multirow}

\renewcommand{\dateseparator}{-}

\setlength{\parskip}{1em}
\hyphenation{Er-ror-Res-pon-se}

%% Special references
\newcommand{\fref}[1]{{\textcolor{ArrowheadBlue}{\hyperref[sec:functions:#1]{#1}}}}
\newcommand{\mref}[1]{{\textcolor{ArrowheadPurple}{\hyperref[sec:model:#1]{#1}}}}
\newcommand{\prref}[1]{{\textcolor{ArrowheadPurple}{\hyperref[sec:model:primitives:#1]{#1}}}}
\newcommand{\pdef}[1]{{\textcolor{ArrowheadGrey}{#1\label{sec:model:primitives:#1}\label{sec:model:primitives:#1s}\label{sec:model:primitives:#1es}}}}
\newcommand{\pref}[1]{{\textcolor{ArrowheadGrey}{\hyperref[sec:model:primitives:#1]{#1}}}}

\newrobustcmd\fsubsection[5]{
  \addtocounter{subsection}{1}
  \addcontentsline{toc}{subsection}{\protect\numberline{\thesubsection}operation \textcolor{ArrowheadBlue}{#1}}
  \renewcommand*{\do}[1]{\rref{##1},\ }
  \subsection*{
    \thesubsection\quad
    operation
    \textcolor{ArrowheadBlue}{#1}
    (\notblank{#2}{\mref{#2}}{})
    \notblank{#3}{: \mref{#3}}{}
    \notblank{#4}{: \prref{#4}}{}
    \notblank{#5}{/ \mref{#5}}{}
  }
  \label{sec:functions:#1}
}
\newrobustcmd\msubsection[2]{
  \addtocounter{subsection}{1}
  \addcontentsline{toc}{subsection}{\protect\numberline{\thesubsection}#1 \textcolor{ArrowheadPurple}{#2}}
  \subsection*{\thesubsection\quad#1 \textcolor{ArrowheadPurple}{#2}}
  \label{sec:model:#2} \label{sec:model:#2s} \label{sec:model:#2es}
}
\newrobustcmd\msubsubsection[3]{
  \addtocounter{subsubsection}{1}
  \addcontentsline{toc}{subsubsection}{\protect\numberline{\thesubsubsection}#1 \textcolor{ArrowheadPurple}{#2}}
  \subsubsection*{\thesubsubsection\quad#1 \textcolor{ArrowheadPurple}{#2}}
  \label{sec:model:#2} \label{sec:model:#2s}
}
%%

\begin{document}

%% Arrowhead Document Properties
\ArrowheadTitle{blacklistManagement} % XXX = ServiceName 
\ArrowheadServiceID{blacklistManagement} % ID name of service
\ArrowheadType{Service Description}
\ArrowheadTypeShort{SD}
\ArrowheadVersion{5.0.0} % Arrowhead version X.Y.Z, e..g. 4.4.1
\ArrowheadDate{\today}
\ArrowheadAuthor{Katinka Jakó} % Corresponding author e.g. Jerker Delsing
\ArrowheadStatus{DRAFT} % e..g. RELEASE, RELEASE CONDIDATE, PROTOTYPE
\ArrowheadContact{jako.katinka@aitia.ai} % Email of corresponding author
\ArrowheadFooter{\href{www.arrowhead.eu}{www.arrowhead.eu}}
\ArrowheadSetup
%%

%% Front Page
\begin{center}
  \vspace*{1cm}
  \huge{\arrowtitle}

  \vspace*{0.2cm}
  \LARGE{\arrowtype}
  \vspace*{1cm}

  %\Large{Service ID: \textit{"\arrowid"}}
  \vspace*{\fill}

  % Front Page Image
  %\includegraphics{figures/TODO}

  \vspace*{1cm}
  \vspace*{\fill}

  % Front Page Abstract
  \begin{abstract}
    This document provides service description for the \textbf{blacklistManagement} service. 
  \end{abstract}

  \vspace*{1cm}

%   \scriptsize
%   \begin{tabularx}{\textwidth}{l X}
%     \raisebox{-0.5\height}{\includegraphics[width=2cm]{figures/artemis_logo}} & {ARTEMIS Innovation Pilot Project: Arrowhead\newline
%     THEME [SP1-JTI-ARTEMIS-2012-AIPP4 SP1-JTI-ARTEMIS-2012-AIPP6]\newline
%     [Production and Energy System Automation Intelligent-Built environment and urban infrastructure for sustainable and friendly cities]}
%   \end{tabularx}
%   \vspace*{-0.2cm}
 \end{center}

\newpage
%%

%% Table of Contents
\tableofcontents
\newpage
%%

\section{Overview}
\label{sec:overview}
This document describes the \textbf{blacklistManagement} service, which enables systems (with operator role or proper permissions) to handle (query, create, remove) blacklist entries in bulk. An example of this interaction is when an operator bans a consumer because it is flooding the Local Cloud with thousands of requests. To enable other systems to use, to consume it, this service needs to be offered through the ServiceRegistry.

The \textbf{blacklistManagement} service contains the following operations:

\begin{itemize}
    \item \textit{query} lists the entries that match the filtering requirements;
    \item \textit{create} adds the specified entries to the blacklist;
    \item \textit{remove} deactivates the entries that apply to the systems with the specified names.
\end{itemize}

The rest of this document is organized as follows.
In Section \ref{sec:functions}, we describe the abstract message operations provided by the service.
In Section \ref{sec:model}, we end the document by presenting the data types used by the mentioned operations.

\subsection{How This Service Is Meant to Be Used}
The service's purpose is to handle the blacklist entries centrally and in bulk. Thus, the \textbf{blacklistManagement} service makes it possible to manage who is (not) banned from the Local Cloud.

Application systems should not use this service, only human operators (indirectly via some tool, for example) or dedicated Support systems.

\subsection{Important Delimitations}
\label{sec:delimitations}

The requester has to identify itself to use any of the operations.

\subsection{Access policy}
\label{sec:accesspolicy}

The service is only available for operators, dedicated Support systems and those who have the proper authorization rights to consume it.

\newpage

\section{Service Operations}
\label{sec:functions}

This section describes the abstract signatures of each operations of the service. The \textbf{blacklistManagement} service is used to \textit{query}, \textit{create} and \textit{remove} blacklist entries.
In particular, each subsection names an operation, an input type and one or two output types (unsuccessful operations can return different structure), in that order.
The input type is named inside parentheses, while the output type is preceded by a colon. If the operation has two output types, they are separated by a slash.
Input and output types are only denoted when accepted or returned, respectively, by the operation in question. All abstract data types named in this section are defined in Section 3.

\phantomsection
\fsubsection{query}{BlacklistQueryRequest}{BlacklistEntryListResponse}{}{ErrorResponse}

Operation \textit{query} lists the blacklist entries that match the filtering requirements. The query data must meet the following criteria:

\begin{itemize}
    \item The operation returns results in pages. There are default page data settings, but the requester can provide a custom specification.
    \item If page number is specified, the page size must be specified as well and vice versa.
    \item In some Local Clouds there is a maximum page size.
    \item If a filter expects a list, there is an OR relation between the elements of the filter.
    \item There is an AND relation between different kind of filters.
    \item If \textit{alivesAt} is set, inactive records will not be returned.
\end{itemize}

\fsubsection{create}{BlacklistCreateListRequest}{BlacklistEntryListResponse}{}{ErrorResponse}

Operation \textit{create} creates the specified blacklist entries. The creation data must meet the following criteria:

\begin{itemize}
    \item System names are case sensitive, must follow the PascalCase naming convention and have to be unique within the Local Cloud.
    \item System names can contain maximum 63 character of letters (English alphabet), and numbers, and have to start with a letter.
    \item The expiration date has to be in the future.
    \item Reason must always be specified and can contain maximum 1024 characters.
\end{itemize}

\fsubsection{remove}{BlacklistRemoveRequest}{OperationStatus}{}{ErrorResponse}

Operation \textit{remove} deactivates the entries that apply to the specified system names. The entries will stay in the database and it is possible to query them later as inactive records.

\clearpage

\section{Information Model}
\label{sec:model}

Here, all data objects that can be part of the \textbf{blacklistManagement} service are listed and must be respected by the hosting system.
Note that each subsection, which describes one type of object, begins with the \textit{struct} keyword, which is used to denote a collection of named fields, each with its own data type.
As a complement to the explicitly defined types in this section, there is also a list of implicit primitive types in Section \ref{sec:model:primitives}, which are used to represent things like hashes and identifiers.

\phantomsection

\msubsection{struct}{BlacklistQueryRequest}
 
\begin{table}[ht!]
\begin{tabularx}{\textwidth}{| p{3cm} | p{3.5cm} | p{1.75cm} | X |} \hline
\rowcolor{gray!33} Field & Type & Mandatory & Description \\ \hline
authentication & \hyperref[sec:model:Identity]{Identity} & yes & The requester of the operation. \\ \hline
pageNumber & \pref{Number} & no (yes) & The number of the requested page. It is mandatory, if page size is specified. \\ \hline
pageSize & \pref{Number} & no (yes) & The number of entries on the requested page. It is mandatory, if page number is specified. \\ \hline
pageSortField & \pref{String} & no & The identifier of the field which must be used to sort the entries. \\ \hline
pageDirection & \pref{Direction} & no & The direction of the sorting. \\ \hline
systemNames & \pref{List}$<$\pref{SystemName}$>$ & no & Requester is looking for blacklist entries that apply to systems with any of the specified names. \\ \hline
mode & \pref{Mode} & no & Requester is looking for blacklist entries with the specified activity. \\ \hline
issuers & \pref{List}$<$\pref{SystemName}$>$ & no & Requester is looking for blacklist entries that were created by systems with any of the specified names. \\ \hline
revokers & \pref{List}$<$\pref{SystemName}$>$ & no & Requester is looking for blacklist entries that were revoked by systems with any of the specified names. \\ \hline
reason & \pref{String} & no & Requester is looking for blacklist entries that were created for the specified reason or the reason contains this text. \\ \hline
alivesAt & \pref{DateTime} & no & Requester is looking for active blacklist records that are not expired at this timestamp. \\ \hline
\end{tabularx}
\end{table}

\msubsection{struct}{Identity}

An \pref{Object} which describes the identity of a system. It also contains whether the identified system has higher level administrative rights.

\clearpage

\msubsection{struct}{BlacklistEntryListResponse}
 
\begin{table}[ht!]
\begin{tabularx}{\textwidth}{| p{4.25cm} | p{5cm} | X |} \hline
\rowcolor{gray!33} Field & Type      & Description \\ \hline
status & \pref{OperationStatus} & Status of the operation. \\ \hline
entries & \pref{List}$<$\hyperref[sec:model:BlacklistEntryResponse]{BlacklistEntryResponse}$>$ & List of blacklist entry results. \\ \hline
count & \pref{Number} & The total number of corresponding entries. \\ \hline
\end{tabularx}
\end{table}

\msubsection{struct}{BlacklistEntryResponse}

\begin{table}[ht!]
\begin{tabularx}{\textwidth}{| p{4.25cm} | p{3.5cm} | X |} \hline
\rowcolor{gray!33} Field & Type      & Description \\ \hline
systemName & \pref{SystemName} & Unique identifier of the blacklisted system. \\ \hline
createdBy & \pref{SystemName} & Unique identifier of the system that created the record. \\ \hline
revokedBy & \pref{SystemName} & Unique identifier of the system that revoked the record. Only appears if the record was revoked. \\ \hline
createdAt & \pref{DateTime} & Blacklist record was created at this timestamp. \\ \hline
updatedAt & \pref{DateTime} & Blacklist record was updated at this timestamp. \\ \hline
reason & \pref{String} & The system was blacklisted because of this reason. \\ \hline
expiresAt & \pref{DateTime} & Blacklist record expires at this timestamp. Only appears if the record can expire. \\ \hline
active & \pref{Boolean} & Indicates if the rule defined by the entry is active. Only false if the rule has been explicitly revoked. \\ \hline

\end{tabularx}
\end{table}

\msubsection{struct}{ErrorResponse}

\begin{table}[ht!]
\begin{tabularx}{\textwidth}{| p{4.25cm} | p{3.5cm} | X |} \hline
\rowcolor{gray!33} Field & Type      & Description \\ \hline
status & \pref{OperationStatus} & Status of the operation. \\ \hline
errorMessage & \pref{String} & Description of the error. \\ \hline
errorCode &\pref{Number}  & Numerical code of the error. \\ \hline
type & \pref{ErrorType} & Type of the error. \\ \hline
origin & \pref{String} & Origin of the error. \\ \hline
\end{tabularx}
\end{table}

\clearpage

\msubsection{struct}{BlacklistCreateListRequest}
 
\begin{table}[ht!]
\begin{tabularx}{\textwidth}{| p{3cm} | p{5cm} | p{2cm} | X |} \hline
\rowcolor{gray!33} Field & Type & Mandatory & Description \\ \hline
authentication & \hyperref[sec:model:Identity]{Identity} & yes & The requester of the operation. \\ \hline
entries & \pref{List}$<$\hyperref[sec:model:BlacklistCreateRequest]{BlacklistCreateRequest}$>$ & yes & List of blacklist entries to create. \\ \hline
\end{tabularx}
\end{table}

\msubsection{struct}{BlacklistCreateRequest}
 
\begin{table}[ht!]
\begin{tabularx}{\textwidth}{| p{3cm} | p{5cm} | p{2cm} | X |} \hline
\rowcolor{gray!33} Field & Type & Mandatory & Description \\ \hline
systemName & \pref{SystemName} & yes & The name of the system to be blacklisted. \\ \hline
expiresAt & \pref{DateTime} & no & This rule will expire at this timestamp. \\ \hline
reason & \pref{String} & yes & The system is blacklisted because of this reason. \\ \hline
\end{tabularx}
\end{table}

\msubsection{struct}{BlacklistRemoveRequest}
 
\begin{table}[ht!]
\begin{tabularx}{\textwidth}{| p{3cm} | p{3.5cm} | p{1.75cm} | X |} \hline
\rowcolor{gray!33} Field & Type & Mandatory & Description \\ \hline
authentication & \hyperref[sec:model:Identity]{Identity} & yes & The requester of the operation. \\ \hline
systemNames & \pref{List}$<$\pref{SystemName}$>$ & yes & The names of the systems to remove from the blacklist. \\ \hline
\end{tabularx}
\end{table}

\clearpage

\subsection{Primitives}
\label{sec:model:primitives}

Types and structures mentioned throughout this document that are assumed to be available to implementations of this service.
The concrete interpretations of each of these types and structures must be provided by any IDD document claiming to implement this service.


\begin{table}[ht!]
\begin{tabularx}{\textwidth}{| p{3cm} | X |} \hline
\rowcolor{gray!33} Type & Description \\ \hline
\pdef{Boolean}          & One out of true or false. \\ \hline
\pdef{DateTime}         & Pinpoints a specific moment in time. \\ \hline
\pdef{Direction}        & The direction of a sorting operation. Possible values are the representation of ascending or descending order. \\ \hline
\pdef{ErrorType}        & Any suitable type chosen by the implementor of service. \\ \hline
\pdef{List}$<$A$>$      & An \textit{array} of a known number of items, each having type A. \\ \hline
\pdef{Mode}             & Specifies whether the queried records should have the \textit{active} flag set. The possible values are: ALL, ACTIVES, INACTIVES. \\ \hline
\pdef{Number}           & Decimal number. \\ \hline
\pdef{Object}           & Set of primitives and possible further objects. \\ \hline
\pdef{OperationStatus}  & Logical, textual or numerical value that indicates whether an operation is a success or a failure. Multiple values can be used for success and error cases to give additional information about the nature of the result. \\ \hline
\pdef{String}           & A chain of characters. \\ \hline
\pdef{SystemName}       & A string identifier that is intended to be both human and machine-readable. Must follow PascalCase naming convention. \\ \hline
\end{tabularx}
\end{table}

\newpage

\bibliographystyle{IEEEtran}
\bibliography{bibliography}

\newpage

\section{Revision History}
\subsection{Amendments}

\noindent\begin{tabularx}{\textwidth}{| p{1cm} | p{3cm} | p{2cm} | X | p{4cm} |} \hline
\rowcolor{gray!33} No. & Date & Version & Subject of Amendments & Author \\ \hline

1 & YYYY-MM-DD & \arrowversion & & Xxx Yyy \\ \hline
\end{tabularx}

\subsection{Quality Assurance}

\noindent\begin{tabularx}{\textwidth}{| p{1cm} | p{3cm} | p{2cm} | X |} \hline
\rowcolor{gray!33} No. & Date & Version & Approved by \\ \hline

1 & YYYY-MM-DD & \arrowversion  &  \\ \hline

\end{tabularx}

\end{document}