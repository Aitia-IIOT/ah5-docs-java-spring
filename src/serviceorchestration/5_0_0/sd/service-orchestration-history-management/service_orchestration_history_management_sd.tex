\documentclass[a4paper]{arrowhead}

\usepackage[yyyymmdd]{datetime}
\usepackage{etoolbox}
\usepackage[utf8]{inputenc}
\usepackage{multirow}

\renewcommand{\dateseparator}{-}

\setlength{\parskip}{1em}
\hyphenation{Er-ror-Res-pon-se}

%% Special references
\newcommand{\fref}[1]{{\textcolor{ArrowheadBlue}{\hyperref[sec:functions:#1]{#1}}}}
\newcommand{\mref}[1]{{\textcolor{ArrowheadPurple}{\hyperref[sec:model:#1]{#1}}}}
\newcommand{\prref}[1]{{\textcolor{ArrowheadPurple}{\hyperref[sec:model:primitives:#1]{#1}}}}
\newcommand{\pdef}[1]{{\textcolor{ArrowheadGrey}{#1\label{sec:model:primitives:#1}\label{sec:model:primitives:#1s}\label{sec:model:primitives:#1es}}}}
\newcommand{\pref}[1]{{\textcolor{ArrowheadGrey}{\hyperref[sec:model:primitives:#1]{#1}}}}

\newrobustcmd\fsubsection[5]{
  \addtocounter{subsection}{1}
  \addcontentsline{toc}{subsection}{\protect\numberline{\thesubsection}operation \textcolor{ArrowheadBlue}{#1}}
  \renewcommand*{\do}[1]{\rref{##1},\ }
  \subsection*{
    \thesubsection\quad
    operation
    \textcolor{ArrowheadBlue}{#1}
    (\notblank{#2}{\mref{#2}}{})
    \notblank{#3}{: \mref{#3}}{}
    \notblank{#4}{: \prref{#4}}{}
    \notblank{#5}{/ \mref{#5}}{}
  }
  \label{sec:functions:#1}
}
\newrobustcmd\msubsection[2]{
  \addtocounter{subsection}{1}
  \addcontentsline{toc}{subsection}{\protect\numberline{\thesubsection}#1 \textcolor{ArrowheadPurple}{#2}}
  \subsection*{\thesubsection\quad#1 \textcolor{ArrowheadPurple}{#2}}
  \label{sec:model:#2} \label{sec:model:#2s} \label{sec:model:#2es}
}
\newrobustcmd\msubsubsection[3]{
  \addtocounter{subsubsection}{1}
  \addcontentsline{toc}{subsubsection}{\protect\numberline{\thesubsubsection}#1 \textcolor{ArrowheadPurple}{#2}}
  \subsubsection*{\thesubsubsection\quad#1 \textcolor{ArrowheadPurple}{#2}}
  \label{sec:model:#2} \label{sec:model:#2s}
}
%%

\begin{document}

%% Arrowhead Document Properties
\ArrowheadTitle{serviceOrchestrationHistoryManagement} % XXX = ServiceName 
\ArrowheadServiceID{serviceOrchestrationHistoryManagement} % ID name of service
\ArrowheadType{Service Description}
\ArrowheadTypeShort{SD}
\ArrowheadVersion{5.0.0} % Arrowhead version X.Y.Z, e..g. 4.4.1
\ArrowheadDate{\today}
\ArrowheadAuthor{Tamás Bordi} % Corresponding author e.g. Jerker Delsing
\ArrowheadStatus{DRAFT} % e..g. RELEASE, RELEASE CONDIDATE, PROTOTYPE
\ArrowheadContact{tbordi@aitia.ai} % Email of corresponding author
\ArrowheadFooter{\href{www.arrowhead.eu}{www.arrowhead.eu}}
\ArrowheadSetup
%%

%% Front Page
\begin{center}
  \vspace*{1cm}
  \huge{\arrowtitle}

  \vspace*{0.2cm}
  \LARGE{\arrowtype}
  \vspace*{1cm}

  %\Large{Service ID: \textit{"\arrowid"}}
  \vspace*{\fill}

  % Front Page Image
  %\includegraphics{figures/TODO}

  \vspace*{1cm}
  \vspace*{\fill}

  % Front Page Abstract
  \begin{abstract}
    This document provides service description for the \textbf{serviceOrchestrationHistoryManagement} service. 
  \end{abstract}

  \vspace*{1cm}

%   \scriptsize
%   \begin{tabularx}{\textwidth}{l X}
%     \raisebox{-0.5\height}{\includegraphics[width=2cm]{figures/artemis_logo}} & {ARTEMIS Innovation Pilot Project: Arrowhead\newline
%     THEME [SP1-JTI-ARTEMIS-2012-AIPP4 SP1-JTI-ARTEMIS-2012-AIPP6]\newline
%     [Production and Energy System Automation Intelligent-Built environment and urban infrastructure for sustainable and friendly cities]}
%   \end{tabularx}
%   \vspace*{-0.2cm}
 \end{center}

\newpage
%%

%% Table of Contents
\tableofcontents
\newpage
%%

\section{Overview}
\label{sec:overview}
This document describes the \textbf{serviceOrchestrationHistoryManagement} service, which enables systems (with operator role or proper permissions) to query the service orchestration job details. An example of this interaction is that a higher entity (a dedicated system directly or a human operator indirectly via some tool) with management access queries the orchestration history in order to verify whether a specific orchestration process was executed properly or not. 

Service orchestration job records are created and updated during every single execution of the orchestration process (see \textit{serviceOrchestration} and \textit{serviceOrchestrationPushManagement} service descriptions) and storing status and other kind of relevant information about the given orchestration processes.

The \textbf{serviceOrchestrationHistoryManagement} service contains the following operations:

\begin{itemize}
    \item \textit{query} lists the orchestration history records that match the filtering requirements;
\end{itemize}

The rest of this document is organized as follows.
In Section \ref{sec:functions}, we describe the abstract message operations provided by the service.
In Section \ref{sec:model}, we end the document by presenting the data types used by the mentioned operations.

\subsection{How This Service Is Meant to Be Used}

The purpose of the service is to list the stored orchestration jobs according to given filtering requirements.

The most common use case is when a higher entity has triggered some push orchestration and wants to verify that all of them were executed without any issues and matching service instances were orchestrated and successfully pushed to the consumer systems.

\subsection{Important Delimitations}
\label{sec:delimitations}

The requester has to identify itself to use any of the operations. 

\subsection{Access policy}
\label{sec:accesspolicy}

The service is only available for operators, dedicated Core/Support systems and those who have the proper authorization rights to consume it.

\newpage

\section{Service Operations}
\label{sec:functions}

This section describes the abstract signatures of each operations of the service. The \textbf{serviceOrchestrationHistoryManagement} service is used to \textit{query} the orchestration job records.
In particular, each subsection names an operation, an input type and one or two output types (unsuccessful operations can return different structure), in that order.
The input type is named inside parentheses, while the output type is preceded by a colon. If the operation has two output types, they are separated by a slash.
Input and output types are only denoted when accepted or returned, respectively, by the operation in question. All abstract data types named in this section are defined in Section 3.

\phantomsection

\fsubsection{query}{OrchestrationHistoryQueryRequest}{OrchestrationHistoryResponse}{}{ErrorResponse}

Operation \textit{query} lists the orchestration job records that match the filtering requirements. The query data must meet the following criteria:

\begin{itemize}
    \item The operation returns results in pages. There are default page data settings, but the requester can provide a custom specification.
    \item If page number is specified, the page size must be specified as well and vice versa.
    \item In some Local Clouds there is a maximum page size.
    \item If a filter expects a list, there is an OR relation between the elements of the filter.
    \item There is an AND relation between different kind of filters.
\end{itemize}

\clearpage

\section{Information Model}
\label{sec:model}

Here, all data objects that can be part of the \textbf{serviceOrchestrationLockManagement} service are listed and must be respected by the hosting system.
Note that each subsection, which describes one type of object, begins with the \textit{struct} keyword, which is used to denote a collection of named fields, each with its own data type.
As a complement to the explicitly defined types in this section, there is also a list of implicit primitive types in Section \ref{sec:model:primitives}, which are used to represent things like hashes and identifiers.

\phantomsection

\msubsection{struct}{OrchestrationHistoryQueryRequest}

\begin{table}[ht!]
\begin{tabularx}{\textwidth}{| p{3cm} | p{5.5cm} | p{2cm} | X |} \hline
\rowcolor{gray!33} Field & Type & Mandatory & Description \\ \hline
authentication & \hyperref[sec:model:Identity]{Identity} & yes & The requester of the operation. \\ \hline
pageNumber & \pref{Number} & no (yes) & The number of the requested page. It is mandatory, if page size is specified. \\ \hline
pageSize & \pref{Number} & no (yes) & The number of entries on the requested page. It is mandatory, if page number is specified. \\ \hline
pageSortField & \pref{String} & no & The identifier of the field which must be used to sort the entries. \\ \hline
pageDirection & \pref{Direction} & no & The direction of the sorting. \\ \hline
ids & \pref{List}$<$\pref{OrchestrationJobId}$>$ & no &  Requester is looking for job records with any of the specified job identifiers. \\ \hline
statuses & \pref{List}$<$\pref{OrchestrationJobStatus}$>$ & no &  Requester is looking for job records with any of the specified job statuses. \\ \hline
type & \pref{OrchestrationType} & no &  Requester is looking for job records with the specified type. \\ \hline
requesterSystems & \pref{List}$<$\pref{SystemName}$>$ & no & Requester is looking for job records with any of the specified requester systems. \\ \hline
taregtSystems & \pref{List}$<$\pref{SystemName}$>$ & no & Requester is looking for job records with any of the specified target systems. \\ \hline
serviceDefinitions & \pref{List}$<$\pref{ServciceName}$>$ & no & Requester is looking for job records with any of the specified service definitions. \\ \hline
subscriptionIds & \pref{List}$<$\pref{OrchestrationSubscriptionId}$>$ & no &  Requester is looking for job records with any of the specified subscription identifiers. \\ \hline
\end{tabularx}
\end{table}

\msubsection{struct}{Identity}

An \pref{Object} which describes the identity of a system. It also contains whether the identified system has higher level administrative rights.

\msubsection{struct}{OrchestrationHistoryResponse}

\begin{table}[ht!]
\begin{tabularx}{\textwidth}{| p{3cm} | p{6.5cm} | X |} \hline
\rowcolor{gray!33} Field & Type & Description \\ \hline
status & \pref{OperationStatus} & Status of the operation. \\ \hline
entries & \pref{List}$<$\hyperref[sec:model:OrchestrationJobResponse]{OrchestrationJobResponse}$>$ & List of orchestration job records. \\ \hline
count & \pref{Number} & Total number of orchestration job records. \\ \hline
\end{tabularx}
\end{table}

\msubsection{struct}{OrchestrationJobResponse}

\begin{table}[ht!]
\begin{tabularx}{\textwidth}{| p{2.8cm} | p{4.5cm} | X |} \hline
\rowcolor{gray!33} Field & Type & Description \\ \hline
id & \pref{OrchestrationJobId} & Unique job identifier. \\ \hline
status & \pref{OrchestrationJobStatus} & Actual working state of the job. \\ \hline
type & \pref{OrchestrationType} & Type of orchestration. \\ \hline
requesterSystem & \pref{SystemName} & Name of the system that started the orchestration process. \\ \hline
targetSystem & \pref{SystemName} & Name of the system for which the orchestration is executed. \\ \hline
serviceDefinition & \pref{ServiceName} & Name of the service that the orchestration job is targeting. \\ \hline
subscriptionId & \pref{OrchestrationSubscriptionId} & Unique identifier of associated subscription record. \\ \hline
message & \pref{String} & Additional error or warning information. \\ \hline
createdAt & \pref{DateTime} & The job was created at this timestamp. \\ \hline
startedAt & \pref{DateTime} & The job was started at this timestamp. \\ \hline
finishedAt & \pref{DateTime} & The job was finished at this timestamp. \\ \hline
\end{tabularx}
\end{table}

\msubsection{struct}{ErrorResponse}

\begin{table}[ht!]
\begin{tabularx}{\textwidth}{| p{4.25cm} | p{3.5cm} | X |} \hline
\rowcolor{gray!33} Field & Type      & Description \\ \hline
status & \pref{OperationStatus} & Status of the operation. \\ \hline
errorMessage & \pref{String} & Description of the error. \\ \hline
errorCode &\pref{Number}  & Numerical code of the error. \\ \hline
type & \pref{ErrorType} & Type of the error. \\ \hline
origin & \pref{String} & Origin of the error. \\ \hline
\end{tabularx}
\end{table}

\clearpage

\subsection{Primitives}
\label{sec:model:primitives}

Types and structures mentioned throughout this document that are assumed to be available to implementations of this service.
The concrete interpretations of each of these types and structures must be provided by any IDD document claiming to implement this service.


\begin{table}[ht!]
\begin{tabularx}{\textwidth}{| p{4.3cm} | X |} \hline
\rowcolor{gray!33} Type & Description \\ \hline
\pdef{DateTime}         & Pinpoints a specific moment in time. \\ \hline
\pdef{Direction}        & The direction of a sorting operation. Possible values are the representation of ascending or descending order. \\ \hline
\pdef{ErrorType}        & Any suitable type chosen by the implementor of service. \\ \hline
\pdef{List}$<$A$>$      & An \textit{array} of a known number of items, each having type A. \\ \hline
\pdef{Number}           & Decimal number. \\ \hline
\pdef{Object}           & Set of primitives and possible further objects. \\ \hline
\pdef{OperationStatus}  & Logical, textual or numerical value that indicates whether an operation is a success or a failure. Multiple values can be used for success and error cases to give additional information about the nature of the result. \\ \hline
\pdef{OrchestrationJobId} & Unique string identifier. \\ \hline
\pdef{OrchestrationJobStatus} & Predefined values indicating working states. \\ \hline
\pdef{OrchestrationSubscriptionId} & Unique string identifier.\\ \hline
\pdef{OrchestrationType} & Predefined values indicating orchestration type (pull or push). \\ \hline
\pdef{ServiceName} & A string identifier that is intended to be both human and machine-readable. Must follow camelCase naming convention. \\ \hline
\pdef{String}           & A chain of characters. \\ \hline
\pdef{SystemName}             & A string identifier that is intended to be both human and machine-readable. Must follow PascalCase naming convention. \\ \hline
\end{tabularx}
\end{table}

\newpage

\bibliographystyle{IEEEtran}
\bibliography{bibliography}

\newpage

\section{Revision History}
\subsection{Amendments}

\noindent\begin{tabularx}{\textwidth}{| p{1cm} | p{3cm} | p{2cm} | X | p{4cm} |} \hline
\rowcolor{gray!33} No. & Date & Version & Subject of Amendments & Author \\ \hline

1 & YYYY-MM-DD & \arrowversion & & Xxx Yyy \\ \hline
\end{tabularx}

\subsection{Quality Assurance}

\noindent\begin{tabularx}{\textwidth}{| p{1cm} | p{3cm} | p{2cm} | X |} \hline
\rowcolor{gray!33} No. & Date & Version & Approved by \\ \hline

1 & YYYY-MM-DD & \arrowversion  &  \\ \hline

\end{tabularx}

\end{document}