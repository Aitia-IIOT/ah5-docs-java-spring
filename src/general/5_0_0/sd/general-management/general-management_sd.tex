\documentclass[a4paper]{arrowhead}

\usepackage[yyyymmdd]{datetime}
\usepackage{etoolbox}
\usepackage[utf8]{inputenc}
\usepackage{multirow}
\usepackage{float}

\renewcommand{\dateseparator}{-}

\setlength{\parskip}{1em}
\hyphenation{Er-ror-Res-pon-se}
\hyphenation{O-pe-ra-ti-on-Sta-tus}

%% Special references
\newcommand{\fref}[1]{{\textcolor{ArrowheadBlue}{\hyperref[sec:functions:#1]{#1}}}}
\newcommand{\mref}[1]{{\textcolor{ArrowheadPurple}{\hyperref[sec:model:#1]{#1}}}}
\newcommand{\prref}[1]{{\textcolor{ArrowheadPurple}{\hyperref[sec:model:primitives:#1]{#1}}}}
\newcommand{\pdef}[1]{{\textcolor{ArrowheadGrey}{#1\label{sec:model:primitives:#1}\label{sec:model:primitives:#1s}\label{sec:model:primitives:#1es}}}}
\newcommand{\pref}[1]{{\textcolor{ArrowheadGrey}{\hyperref[sec:model:primitives:#1]{#1}}}}

\newrobustcmd\fsubsection[5]{
  \addtocounter{subsection}{1}
  \addcontentsline{toc}{subsection}{\protect\numberline{\thesubsection}operation \textcolor{ArrowheadBlue}{#1}}
  \renewcommand*{\do}[1]{\rref{##1},\ }
  \subsection*{
    \thesubsection\quad
    operation
    \textcolor{ArrowheadBlue}{#1}
    (\notblank{#2}{\mref{#2}}{})
    \notblank{#3}{: \mref{#3}}{}
    \notblank{#4}{: \prref{#4}}{}
    \notblank{#5}{/ \mref{#5}}{}
  }
  \label{sec:functions:#1}
}
\newrobustcmd\msubsection[2]{
  \addtocounter{subsection}{1}
  \addcontentsline{toc}{subsection}{\protect\numberline{\thesubsection}#1 \textcolor{ArrowheadPurple}{#2}}
  \subsection*{\thesubsection\quad#1 \textcolor{ArrowheadPurple}{#2}}
  \label{sec:model:#2} \label{sec:model:#2s} \label{sec:model:#2es}
}
\newrobustcmd\msubsubsection[3]{
  \addtocounter{subsubsection}{1}
  \addcontentsline{toc}{subsubsection}{\protect\numberline{\thesubsubsection}#1 \textcolor{ArrowheadPurple}{#2}}
  \subsubsection*{\thesubsubsection\quad#1 \textcolor{ArrowheadPurple}{#2}}
  \label{sec:model:#2} \label{sec:model:#2s}
}
%%

\begin{document}

%% Arrowhead Document Properties
\ArrowheadTitle{generalManagement} % XXX = ServiceName 
\ArrowheadServiceID{generalManagement} % ID name of service
\ArrowheadType{Service Description}
\ArrowheadTypeShort{SD}
\ArrowheadVersion{5.0.0} % Arrowhead version X.Y.Z, e..g. 4.4.1
\ArrowheadDate{\today}
\ArrowheadAuthor{Rajmund Bocsi} % Corresponding author e.g. Jerker Delsing
\ArrowheadStatus{DRAFT} % e..g. RELEASE, RELEASE CONDIDATE, PROTOTYPE
\ArrowheadContact{rbocsi@aitia.ai} % Email of corresponding author
\ArrowheadFooter{\href{www.arrowhead.eu}{www.arrowhead.eu}}
\ArrowheadSetup
%%

%% Front Page
\begin{center}
  \vspace*{1cm}
  \huge{\arrowtitle}

  \vspace*{0.2cm}
  \LARGE{\arrowtype}
  \vspace*{1cm}

  %\Large{Service ID: \textit{"\arrowid"}}
  \vspace*{\fill}

  % Front Page Image
  %\includegraphics{figures/TODO}

  \vspace*{1cm}
  \vspace*{\fill}

  % Front Page Abstract
  \begin{abstract}
    This document provides a service description for the \textbf{generalManagement} service. 
  \end{abstract}

  \vspace*{1cm}

%   \scriptsize
%   \begin{tabularx}{\textwidth}{l X}
%     \raisebox{-0.5\height}{\includegraphics[width=2cm]{figures/artemis_logo}} & {ARTEMIS Innovation Pilot Project: Arrowhead\newline
%     THEME [SP1-JTI-ARTEMIS-2012-AIPP4 SP1-JTI-ARTEMIS-2012-AIPP6]\newline
%     [Production and Energy System Automation Intelligent-Built environment and urban infrastructure for sustainable and friendly cities]}
%   \end{tabularx}
%   \vspace*{-0.2cm}
 \end{center}

\newpage
%%

%% Table of Contents
\tableofcontents
\newpage
%%

\section{Overview}
\label{sec:overview}
This document describes the \textbf{generalManagement} service, which allows (with operator role or proper permissions) to get some information about the hosting system's behavior, such as log entries and configuration settings. An example of this interaction is when an operator uses the Management Tool to revise the log of a system that behaves abnormally.

The \textbf{generalManagement} service contains the following operations:

\begin{itemize}
    \item \textit{get-log} lists the log entries of the system that matches the filtering requirements;
    \item \textit{get-config} lists the current values of the specified configuration settings;
\end{itemize}

The rest of this document is organized as follows.
In Section \ref{sec:functions}, we describe the abstract message operations provided by the service.
In Section \ref{sec:model}, we end the document by presenting the data types used by the mentioned operations.

\subsection{How This Service Is Meant to Be Used}
The service's purpose is to getting information about a deployed and running system. 

Application systems should not use this service; only operators (via the Management Tool, for example) or dedicated Support systems.

\subsection{Important Delimitations}
\label{sec:delimitations}

The requester has to identify itself to use any of the operations.

\subsection{Access policy}
\label{sec:accesspolicy}

The service is only available for operators, dedicated Support systems and those who have the proper authorization rights to consume it.

\newpage

\section{Service Operations}
\label{sec:functions}

This section describes the abstract signatures of each operation of the service. 
In particular, each subsection names an operation, an input type, and one or two output types (unsuccessful operations can return different structure), in that order.
The input type is named inside parentheses, while the output type is preceded by a colon. If the operation has two output types, they are separated by a slash.
Input and output types are only denoted when accepted or returned, respectively, by the operation in question. All abstract data types named in this section are defined in Section 3.

\phantomsection
\fsubsection{get-log}{LogRequest}{LogResponse}{}{ErrorResponse}

Operation \textit{get-log} lists the log entries of the system that matches the filtering requirements. The input data must meet the following criteria:

\begin{itemize}
    \item The operation returns results in pages. There are default page data settings, but the requester can provide a custom specification.
    \item If page number is specified, the page size must be specified as well and vice versa.
    \item In some Local Clouds, there is a maximum page size.
    \item If both date parameters are specified (\textit{from}, \textit{to}), the date \textit{from} must precede the date \textit{to}.
\end{itemize}

\fsubsection{get-config}{ConfigRequest}{ConfigResponse}{}{ErrorResponse}

Operation  \textit{get-config} lists the current values of the specified configuration settings. The input data must meet the following criteria:

\begin{itemize}
    \item Some configuration settings (e. g. passwords) cannot be acquired with this operation. Such setting names will be ignored.
\end{itemize}

\clearpage

\section{Information Model}
\label{sec:model}

Here, all data objects that can be part of the \textbf{generalManagement} service
are listed and must be respected by the hosting system.
Note that each subsection, which describes one type of object, begins with the \textit{struct} keyword, which is used to denote a collection of named fields, each with its own data type.
As a complement to the explicitly defined types in this section, there is also a list of implicit primitive types in Section \ref{sec:model:primitives}, which are used to represent things like hashes and identifiers.

\phantomsection
\msubsection{struct}{LogRequest}
 
\begin{table}[ht!]
\begin{tabularx}{\textwidth}{| p{2.5cm} | p{2.5cm} | p{2cm} | X |} \hline
\rowcolor{gray!33} Field & Type & Mandatory & Description \\ \hline
authentication & \hyperref[sec:model:Identity]{Identity} & yes & The requester of the operation. \\ \hline
pageNumber & \pref{Number} & no (yes) & The number of the requested page. It is mandatory, if page size is specified. \\ \hline
pageSize & \pref{Number} & no (yes) & The number of entries on the requested page. It is mandatory, if page number is specified. \\ \hline
pageSortField & \pref{String} & no & The identifier of the field which must be used to sort the entries. \\ \hline
pageDirection & \pref{Direction} & no & The direction of the sorting. \\ \hline
from & \pref{DateTime} & no & Requester is looking for log entries with timestamp that does not precede this one. \\ \hline
to & \pref{DateTime} & no & Requester is looking for log entries with timestamp that does not succeed this one. \\ \hline
severity & \pref{LogSeverity} & no & Requester is looking for log entries with severity that equals to or is higher than the specified one. \\ \hline
loggerStr &  \pref{String} & no & Requester is looking for log entries with logger whose name contains the specified text. \\ \hline
\end{tabularx}
\end{table}

\msubsection{struct}{Identity}

An \pref{Object} which describes the identity of a system. It also contains whether the identified system has higher-level administrative rights.

\msubsection{struct}{LogResponse}

\begin{table}[ht!]
\begin{tabularx}{\textwidth}{| p{2.5cm} | p{2.5cm} | X |} \hline
\rowcolor{gray!33} Field & Type      & Description \\ \hline
status & \pref{OperationStatus} & Status of the operation. \\ \hline
entries & \pref{List}$<$\hyperref[sec:model:LogEntry]{LogEntry}$>$ & A page of log entries. \\ \hline
count & \pref{Number} & Total number of entries that match the filters. \\ \hline
\end{tabularx}
\end{table}

\clearpage

\msubsection{struct}{LogEntry}
 
\begin{table}[ht!]
\begin{tabularx}{\textwidth}{| p{2.5cm} | p{2.5cm} | X |} \hline
\rowcolor{gray!33} Field & Type      & Description \\ \hline
logId & \pref{String} & Unique identifier of the entry. \\ \hline
entryDate & \pref{DateTime} & The timestamp of the entry. \\ \hline
logger & \pref{String} & The logger that created the entry. \\ \hline
severity & \pref{LogSeverity} & The severity of the entry. \\ \hline
message & \pref{String} & The log message \\ \hline
exception & \pref{String} & If the log entry is an error, information of the related exception may appear here. \\ \hline
\end{tabularx}
\end{table}

\msubsection{struct}{ErrorResponse}

\begin{table}[ht!]
\begin{tabularx}{\textwidth}{| p{2.5cm} | p{3cm} | X |} \hline
\rowcolor{gray!33} Field & Type      & Description \\ \hline
status & \pref{OperationStatus} & Status of the operation. \\ \hline
errorMessage & \pref{String} & Description of the error. \\ \hline
errorCode &\pref{Number}  & Numerical code of the error. \\ \hline
type & \pref{ErrorType} & Type of the error. \\ \hline
origin & \pref{String} & Origin of the error. \\ \hline
\end{tabularx}
\end{table}

\msubsection{struct}{ConfigRequest}
 
\begin{table}[ht!]
\begin{tabularx}{\textwidth}{| p{2.5cm} | p{2.5cm} | p{2cm} | X |} \hline
\rowcolor{gray!33} Field & Type & Mandatory & Description \\ \hline
authentication & \hyperref[sec:model:Identity]{Identity} & yes & The requester of the operation. \\ \hline
keys & \pref{List}$<$\pref{Name}$>$ & yes & The names of the requested configuration settings. \\ \hline
\end{tabularx}
\end{table}

\msubsection{struct}{ConfigResponse}

\begin{table}[ht!]
\begin{tabularx}{\textwidth}{| p{2.5cm} | p{2.5cm} | X |} \hline
\rowcolor{gray!33} Field & Type      & Description \\ \hline
status & \pref{OperationStatus} & Status of the operation. \\ \hline
map & \hyperref[sec:model:ConfigMap]{ConfigMap} & The actual values of the requested settings. \\ \hline
\end{tabularx}
\end{table}

\msubsection{struct}{ConfigMap}

An \pref{Object} which maps \pref{String} keys to \pref{String} values.

\clearpage

\subsection{Primitives}
\label{sec:model:primitives}

Types and structures mentioned throughout this document that are assumed to be available to implementations of this service.
Concrete interpretations of each of these types and structures must be provided in any IDD document that claims to implement this service.

\begin{table}[ht!]
\begin{tabularx}{\textwidth}{| p{3cm} | X |} \hline
\rowcolor{gray!33} Type & Description \\ \hline
\pdef{DateTime}         & Pinpoints a specific moment in time. \\ \hline
\pdef{Direction}        & The direction of a sorting operation. Possible values are the representation of ascending or descending order. \\ \hline
\pdef{ErrorType}        & Any suitable type chosen by the implementor of service. \\ \hline
\pdef{List}$<$A$>$      & An \textit{array} of a known number of items, each having type A. \\ \hline
\pdef{LogSeverity}      & The kind and seriousness of a log message. \\ \hline
\pdef{Name}             & A string identifier that is intended to be both human and machine-readable. \\ \hline
\pdef{Number}           & Decimal number. \\ \hline
\pdef{Object}           & Set of primitives and possible further objects. \\ \hline
\pdef{OperationStatus}  & Logical, textual or numerical value that indicates whether an operation is a success or a failure. Multiple values can be used for success and error cases to give additional information about the nature of the result. \\ \hline
\pdef{String}           & A chain of characters. \\ \hline
\end{tabularx}
\end{table}

\newpage

\bibliographystyle{IEEEtran}
\bibliography{bibliography}

\newpage

\section{Revision History}
\subsection{Amendments}

\noindent\begin{tabularx}{\textwidth}{| p{1cm} | p{3cm} | p{2cm} | X | p{4cm} |} \hline
\rowcolor{gray!33} No. & Date & Version & Subject of Amendments & Author \\ \hline

1 & YYYY-MM-DD & \arrowversion & & Xxx Yyy \\ \hline
\end{tabularx}

\subsection{Quality Assurance}

\noindent\begin{tabularx}{\textwidth}{| p{1cm} | p{3cm} | p{2cm} | X |} \hline
\rowcolor{gray!33} No. & Date & Version & Approved by \\ \hline

1 & YYYY-MM-DD & \arrowversion  &  \\ \hline

\end{tabularx}

\end{document}